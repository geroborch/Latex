\documentclass[a4paper,10pt]{article}

\usepackage[utf8]{inputenc}
\usepackage[T1]{fontenc}
\usepackage{lmodern}
\usepackage[ngerman]{babel}
\usepackage{hologo}
\usepackage{siunitx}

\begin{document}

\section{Vorlage 239 zur Einsendeaufgabe 3, Kurs 01604}


\subsection{Die Aufgabe}
In dieser Kurseinheit wird \hologo{LaTeX} vorgestellt. \hologo{LaTeX} ist ein \textbf{Textsatzprogramm}, welches das Erstellen von wissenschaftlichen Texten erleichtert. In dieser Aufga-be sollen Sie das vorliegende Dokument möglichst genau mit \hologo{LaTeX} nachbilden. So soll insbesondere die Aufteilung in Abschnitte (hier also in Abschnitt 1, Abschnitt 1.1, usw.), die Formatierungen und der Text dieses Dokumentes \textit{exakt} nachgebildet werden.%


Verwenden Sie die Dokumentenklasse „article“ mit den Optionen „a4paper“ und „10pt“. Setzen Sie das Literaturverzeichnis auf Deutsch, mit \hologo{BibTeX}, analog zu Beispiel 4.6 aus dem Kurstext. Verwenden Sie außerdem das Paket „hologo“ zum Setzen der Logos von \hologo{LaTeX} und \hologo{BibTeX}.%

Für diese Aufgabe gibt es bis zu 40 Punkte.


\subsection {Die Lösung}
Bitte senden Sie folgende Dokumente als Teil der Lösung ein:
\begin{enumerate}
\item die \textbf{original PDF Vorlage} (also diese Datei),
\item das von Ihnen erstellte \textbf{LATEX Dokument} (also die .tex Datei),
\item die von Ihnen erstellte \textbf{Literatur Datenbank} (also die .bib Datei), und
\item  das von Ihnen \textbf{kompilierte Dokument} (also die erzeugte .pdf Datei).
\end{enumerate}
Bitte benennen Sie die Dateien so, dass auf den ersten Blick klar ist, welche Datei welcher der obigen Forderungen entspricht. Auf Leerzeichen und Sonderzeichen wie Umlaute innerhalb der Dateinamen bitten wir zu verzichten.

\section {Tabellen}

Tabelle 1 gibt eine kleine Übersicht über verschiedene Charaktere und Gruppie- rungen in \textit{Der Herr der Ringe.}

\section {Formeln}

In der Informatik spielt die Laufzeitanalyse von Algorithmen eine große Rolle. 
Laufzeiten werden üblicherweise mit den Landau-Symbolen O(f(n)), \si{\ohm}(f(n)), \si{\theta}(f(n)) beschrieben.


\begin{tabular}{||c||l|r||}
\hline 
 Gollum				&Sauron		&Legolas	\\	
\hline
Hobbit								&nein			&nein\\
\hline
Gemeinschaft des Ringes 	&nein 		&ja				\\
\hline
\end{tabular}%

Tabelle 1: Verschiedene Charaktere und Gruppierungen in \textit{Der Herr der Ringe.}





\end{document}
